\chapter{Evaluation}\label{chap:6}


\section{Accuracy evaluation}

In 4436 malwares obtained, 4181 of them are to construct a decision trees. The number of each malware families is shown in Table \ref{table:trainingdata}. Then, 255 remaining malwares meta-data are kept to test experimental result the system.
\begin{table}
  \begin{center}
    \begin{tabular}{ | l | l |}
     \hline
    Malware & Number\\ \hline
    Virut & 2243\\ \hline
	IRCbot & 1169\\ \hline
	Gaobot  & 265\\ \hline
	Autorun & 195\\ \hline
	Downadup &  103\\ \hline
	Mota & 79\\ \hline
	Sality  & 66  \\ \hline
	WaledacMota & 61\\ \hline
    \end{tabular}
	\end{center}
     \caption{Training data.}
      \label{table:trainingdata}
\end{table}

In the decision tree, 255 malwares are taken to check the experimental result of system. 

 The number of each malware families is shown in Table \ref{table:testdata}
 \begin{table}
  \begin{center}
    \begin{tabular}{ | l | l |}
     \hline
    Malware & Number\\ \hline
    Virut & 200\\ \hline
	IRCbot & 100\\ \hline
	Gaobot  & 20\\ \hline
	Autorun & 10\\ \hline
	Downadup &  10\\ \hline
	Mota & 5\\ \hline
	Sality  & 5  \\ \hline
	WaledacMota & 5\\ \hline
    \end{tabular}
	\end{center}
     \caption{Test data.}
      \label{table:testdata}
\end{table}
Table \ref{table:experimentalresult} reveals that result. Some of data in axis show these total number of malware in that family, and that number is separated into groups that these malwares has been classified by the system. For example, there are four malwares in Win32/Virut family, and three of them are successfully sorted into Win32/Virut family and, while one malware is in other family. Therefore, the system recognizes the trojan agent family with 75 \% of accuracy.
\begin{table}
  \begin{center}
    \begin{tabular}{ | l | l | l | l | l | l | l | l | l | l |}
     \hline
    Malware & Virut & IRCbot& Gaobot &  Autorun & Downadup  & Mota & Sality & Waledac & Accuracy\\ \hline
    Virut & 98 & 102 & 0 & 0 & 0 & 0  & 0  & 0 & 49\%\\ \hline
    IRCbot & 5 & 95 & 0 & 0 & 0 & 0  & 0  & 0 & 95\%\\ \hline    
	Gaobot & 0 & 20 & 0 & 0 & 0 & 0  & 0  & 0 & 0\%\\ \hline
	Autorun & 10 & 10 & 0 & 0 & 0 & 0  & 0  & 0 & 0\%\\ \hline	
	Downadup & 0 & 1 & 0 & 9 & 0 & 0  & 0  & 0 & 90\%\\ \hline
	Mota  & 2 & 3 & 50 & 0 & 0 & 0  & 0  & 0 & 0\%\\ \hline
	Sality  & 1 & 1 & 50 & 0 & 1 & 2  & 0  & 0 & 40\%\\ \hline	
	Waledac & 2 & 3 & 50 & 0 & 0 & 0  & 0  & 0 & 0\% \\ \hline
    \end{tabular}
	\end{center}
     \caption{Experimental result}
    \label{table:experimentalresult}
\end{table}
The accuracy of this classification system:
\begin{equation}
Accuracy=\frac{98+95+9+2}{355}=57\%
\end{equation}

The system is useful to help virus researcher determine the malware family that unknown malware belongs to. The malware family contains Win32/Virut, Win32/Autorun, Win32/IRCbot, Win32/Gaobot, Win32/Waledac, Win32/Downadup, Win32/Sality, W32.Mota, known as famous malware family. Virus researchers who know the family of malware can easily find out some semantic similarities between malwares and shows their inner similarity in behavior and static malware characteristics.

\section{Classification efficiency}
Figure shows comparison between steps to detect semantic malware characterization using Virus total with the system.

The figure 1 shows classifying time in the system. The evaluation was performed on a 2.4 HZ core i3 laptop with 4G memory, running in ubuntu 11.4. 	

The result was shown in figure \ref{fig:evaluation1}. The median time to perform classification was 0.25 seconds. The slowest sample that is required 5.12 seconds. Only 6 samples required more than 2 seconds. Processing time of followgraph approach was shown in figure \ref{fig:evaluation2}.

