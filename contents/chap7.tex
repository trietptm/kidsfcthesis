
\chapter{Conclusion}\label{chap:7}
\section{Conclusion}
Analyzing a great number of new malware samples each day is a difficult problem. The fast malware classification system is successful to classify unknown malware into malware families which have semantic specifications. It is strongly believed that system is useful in malware analysis to determine malware behavior and semantic malware characteristics, and to more easily examine a large number of malwares.

However, there is a problem that the system only uses malware meta-data and cannot detect the malware family with same program structure.

Surprisingly, the system cannot be used to classify W32 malware which does not have the signature of PE file signature. 
 
\section{Future work}
The anonymous malwares in the system are classified into malware family, and in the future work, malware shall be automatically  unpacked before classifying malwares. 